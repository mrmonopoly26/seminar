\makeatletter
\def\input@path{{./template/}}
\makeatother
\documentclass[praxis]{template/feif}
\usepackage{tikz}
\usetikzlibrary{positioning}
\usepackage{placeins}
\usepackage{siunitx}
\usepackage{enumitem}
\usepackage{subcaption}
\usepackage{multirow}
\usepackage{mathtools}
\usepackage{makecell}
\usepackage{array}
\usepackage{lipsum}
\usepackage{amsmath}
\usepackage{ifthen} 
\usepackage{hyperref}
\usepackage[T1]{fontenc}
\usepackage[utf8]{inputenc}
\usepackage{layout}
\usepackage{comment}
\usepackage{listings}
\usepackage[inkscapepath=build/svg-inkscape]{svg}
\usepackage[toc, nogroupskip, nonumberlist, nopostdot, acronyms, shortcuts, translate=babel]{glossaries}
\usepackage{tabularray}
\UseTblrLibrary{booktabs}

\newcommand{\rb}[1]{\left(#1\right)}
\newcommand{\bb}[1]{\left[#1\right]}
\newcommand{\cb}[1]{\left\{#1\right\}}
\newcommand{\floor}[1]{\left\lfloor#1\right\rfloor}
\newcommand{\ceil}[1]{\left\lceil#1\right\rceil}
\newcommand{\round}[1]{\left\lfloor#1\right\rceil}
\newcommand{\floorfrac}[2]{\floor{\frac{#1}{#2}}}
\newcommand{\cf}[2]{#1\!\rb{#2}}
\newcommand{\ctf}[2]{\textsc{#1}\!\rb{#2}}
\newcommand{\fmax}[1]{\ctf{Max}{#1}}
\newcommand{\fmin}[1]{\ctf{Min}{#1}}
\newcommand{\ftan}[1]{\ctf{tan}{#1}}
\newcommand{\fsin}[1]{\ctf{sin}{#1}}
\newcommand{\fcos}[1]{\ctf{cos}{#1}}
\newcommand{\fasin}[1]{\ctf{asin}{#1}}
\newcommand{\facos}[1]{\ctf{acos}{#1}}
\newcommand{\abs}[1]{\left|#1\right|}
\newcommand{\Set}[2]{\{\,#1\mid#2\,\}}
\newcommand{\hide}[1]{\iffalse #1 \fi}
\newcommand{\norm}[1]{\left\|\,#1\,\right\|}
\newcommand{\enorm}[1]{\norm{#1}_2}
\newcommand{\nnat}{\mathbb{N}}
\newcommand{\nint}{\mathbb{Z}}
\newcommand{\nrat}{\mathbb{Q}}
\newcommand{\nreal}{\mathbb{R}}
\newcommand{\ncomp}{\mathbb{C}}

\newcommand{\nrealop}{\mathbb{R}_{\geq 0}}
\newcommand{\nrealp}{\mathbb{R}_{>0}}

\DeclareOldFontCommand{\rm}{\normalfont\rmfamily}{\mathrm}
\DeclareOldFontCommand{\sf}{\normalfont\sffamily}{\mathsf}
\DeclareOldFontCommand{\tt}{\normalfont\ttfamily}{\mathtt}
\DeclareOldFontCommand{\bf}{\normalfont\bfseries}{\mathbf}
\DeclareOldFontCommand{\it}{\normalfont\itshape}{\mathit}
\DeclareOldFontCommand{\sl}{\normalfont\slshape}{\@nomath\sl}
\DeclareOldFontCommand{\sc}{\normalfont\scshape}{\@nomath\sc}

\newcommand{\nvec}[1]{\hat{#1}}
\newcommand{\mat}[1]{\mathbf{#1}}
\newcommand{\quat}[1]{\underline{#1}}
\newcommand{\pnt}[1]{\mathbf{#1}}


\def\Titel{Vorlage für verschiedene Dokumente der HS Coburg}
\def\Dozent{<DOZENT>}

% Infos zum Autor
\def\Autorenname{<NACHNAME, VORNAME>}
\def\Geburtsdatum{<GEBURTSDATUM>}
\def\Matrikelnummer{<MATRIKELNUMMER>}
\def\Studiengang{<STUDIENGANG>}

% Infos zum Unternehmen
\def\Unternehmen{<FIRMENNAME>}
\def\Abteilung{<ABTEILUNG>}
\def\Strasse{<STRASSE>}
\def\Ort{<ORT>}

% Infos zum Betreuer
\def\Betreuer{<BETREUER>}
\def\Funktion{<FUNKTION BETREUER>}
\def\Telefon{<TEL. BETREUER>}
\def\Email{<BETREUER EMAIL>}

% Daten
\def\Beginn{<BEGINN>}
\def\Ende{<ENDE>}
\def\Abgabe{\today}

\def\ThesisType{<MASTERARBEIT>}


% Unit-related setup for siunitx
\sisetup{
locale=DE,                   % Use German notation (comma instead of point delimiter for floats)
  per-mode=fraction,           % Switch display to use \frac instead of x^{-1}
  fraction-function=\tfrac,    % Use amsmath's tfrac macro for unit fractions
}

\makeglossaries
%\include{Verzeichnisse/Worttrennungen}
\begin{acronym}[CMAcronyms]
\acro{GPU}{graphics processing unit}
\acro{CPU}{central processing unit}
\acro{ALU}{arithmetic and logical unit}
\acro{API}{application programming interface}
\acro{3D}{three-dimensional}
\acro{2D}{two-dimensional}
\acro{1D}{one-dimensional}
\acro{bpt}{bits per triangle}
\acro{bpv}{bits per vertex}
\acro{LUT}{lookup table}
\acro{SIMD}{single instruction, multiple data}
\acro{AABB}{axis-aligned bounding box}
\acro{GLSL}{OpenGL Shading Language}
\acro{rANS}{range asymmetrical numeral systems}
\acro{POT}{power-of-two}
\acro{LBS}{linear blend skinning}
\acro{DQS}{dual quaternion skinning}
\acro{LMS}{log-matrix skinning}
\acro{SBS}{spherical blend skinning}
\acro{CRS}{centre of rotation-optimized skinning}
\acro{i.i.d.}{independent and identically distributed}
\acro{OSS}{Optimal Simplex Sampling}
\acro{LP}{linear program}
\acro{ILP}{integer linear program}
\acro{MILP}{mixed integer linear program}
\acro{GTS}{generalized triangle strip}
\acro{ATS}{alternating triangle strip}
\acro{CLZ}{count leading zeros}
\acro{TEA}{Tiny Encryption Algorithm}
\acro{ETA}{Enhanced Tunneling Algorithm}
\acro{LW}{Laced Wires}
\acro{Gtps}{Giga triangles per second}
\end{acronym}  
\makenomenclature
\IfFileExists{./lists/symbols.tex} {\nomenclature[0]{\textbf{Symbol}}{\textbf{Bedeutung}}
\nomen{$c$}{Lichtgeschwindigkeit}
\nomen{$h$}{Planksches Wirkungsquantum}
}{}
\IfFileExists{./lists/glossary.tex}{\input{./lists/glossary.tex}}{}
\begin{document}
\maketitle
\frontmatter
% Optional notice to the reader ("Vermerk")
\IfFileExists{./secs/Z0vermerk.tex}{    
  \thispagestyle{empty}
  \input{secs/Z0vermerk}
}{}

\IfFileExists{./secs/A0abstract.tex}{    
  \fancypagestyle{abstract}{
    \fancyhead[L]{Zusammenfassung}
  }
  \phantomsection
  \section*{Zusammenfassung}
  \thispagestyle{abstract}
   \input{./secs/A0abstract}
  }{}  
  \tableofcontents  
  \listoffigures
  \listoftables
  \lstlistoflistings
  \listofsymbols
  \listofacronyms  
  \mainmatter
  % A0abstract.tex is \input'ed in template/main.tex.
% A0teaser.tex is \input'ed in template/main.tex.
\section{Introduction}

Here we test some text.

\noindent Waltz, bad nymph, for quick jigs vex.

\noindent{\rm{Waltz, bad nymph, for quick jigs vex.}}

\noindent{\sf{Waltz, bad nymph, for quick jigs vex.}}

\noindent{\tt{Waltz, bad nymph, for quick jigs vex.}}

\noindent{\bf{Waltz, bad nymph, for quick jigs vex.}}

\noindent{\it{Waltz, bad nymph, for quick jigs vex.}}

\noindent{\sl{Waltz, bad nymph, for quick jigs vex.}}

\noindent{\sc{Waltz, bad nymph, for quick jigs vex.}}

\layout
\section{Previous Work}\label{sec:PreviousWork}
Test some citations~\cite{Alliez03RAC,Peng05T3M,Maglo153MC}.

Here is another citation~\cite{Mlakar24EEC}.

We will show in Section~\ref{Sec:MainPart}, we can fancy stuff.

Write one line per phrase.
Do it like this.

Add a new line to get a new paragraph.
\section{Main Part}\label{Sec:MainPart}

\subsection{Testing some math}

A regular formula

\begin{equation}
    x^2 + y^2 = z^2 
\end{equation}
and another one
\begin{equation}
    \left\Vert \vec{x} \right\Vert = \sqrt{\sum_{i=0}^{n}{x_i^2 + y_i^2}}.
\end{equation}

Here is something with matrices:
\begin{equation}
\bf{A}=
\begin{bmatrix}
    a & b & c\\
    d & e & f\\
\end{bmatrix}^\top.
\end{equation}

An some inline stuff $\sum a_i=\bf{B}$.

\begin{table}
    \caption{Keep a list of symbols in your paper.}
    \label{tab:ListOfSymbols}
    \resizebox{\columnwidth}{!}
    {
    \begin{tblr}{ll}
    \toprule
    \multicolumn{2}{c}{Variables used in this paper}\\
    \midrule
    \bf{Symbol}               & \bf{Meaning}\\
    \midrule
    $\mat{M}$                 & model-view matrix \\
    $\mat{P}$                 & projection matrix \\
    $\nvec{n}$                & a normal vector \\
    $V$                       & number of vertices \\    
    $T$                       & number of triangles \\    
    $\vec{l}$                 & light direction vector \\
    $\pnt{x}$                 & point to be lit \\
    \bottomrule
    \toprule
    \multicolumn{2}{c}{Types used in this paper}\\
    \midrule
    \bf{Notation}               & \bf{Meaning}\\
    \midrule
    $a$                         & scalar\\
    $N$                         & scalar constant\\
    $\alpha$                    & angle\\
    $\vec{x}$                   & column vector\\
    ${x}_i$                     & vector component\\
    $\nvec{x}$                  & normalized vector \\
    $\vec{x}=
    \begin{bmatrix}
        {x}_0\\
        {x}_1\\
        \vdots\\
        {x}_{n-1}
    \end{bmatrix}$              & column vector\\
    $\mat{A}=
    \begin{bmatrix}
        a_{0,0}   & a_{0,1}   & \dots  &  a_{0,m-1}   \\
        a_{1,0}   & a_{1,1}   & \dots  &  a_{1,m-1}   \\
        \vdots    & \vdots    & \ddots &  \vdots      \\
        a_{n-1,0} & a_{n-1,1} & \dots  &  a_{n-1,m-1} \\
    \end{bmatrix}
    $                          & matrix \\
    $\vec{a}_{:,j}$             & $j$-th column of matrix $\mat{A}$\\
    ${a}_{i,j}$                 & element at row $i$, column $j$\\
    $\quat{q}$                  & quaternion \\
    $\pnt{p}$                   & point \\
    $s_i$                       & $i$-th element of a series of scalars\\
    $\vec{x}^{\left[i\right]}$  & $i$-th element of a series of vectors\\
    $\mat{A}^{\left[i\right]}$  & $i$-th element of a series of matrices\\
    $\mathcal{A}$               & set\\
    \bottomrule                                    
    \toprule
    \multicolumn{2}{c}{Operators used in this paper}\\
    \midrule
    \bf{Operation}       & \bf{Meaning}\\
    \midrule
    $\vec{a}^\top$           & transpose, row vector \\
    $\vec{a}^\top\vec{b}$    & inner product \\
    $\vec{a} \times \vec{b}$ & cross product \\    
    \bottomrule                                    
\end{tblr}

    }
\end{table}
\section{Results and Discussion}\label{Sec:Evaluation}

We evaluate the performance of our methods on the meshes of Fig.~\ref{fig:meshes}.
\begin{figure*}
\input{codes/code.tex}%
\caption{Example code.}\label{Code:Python}%
\end{figure*}
    
Hey kids here is Python code in Figure~\ref{Code:Python}.

\begin{figure*}
    \centering%
    \def\svgwidth{\textwidth}%
    \fontsize{6pt}{5pt}\selectfont%
    \includesvg{figs/meshes}%
    \caption{Test Meshes.
    $V$ denotes the number of vertices of the input mesh, $T$ the number of triangles and $M$ the number of meshlets coming from Meshoptimizer.
    Meshlets are visualized with randomized colors.
    }\label{fig:meshes}%
\end{figure*}%

\begin{table}
    \caption{\acs{GTS} comparison for the \textit{Rock} mesh.
    We compare the optimal Gurobi and SCIP solutions against the sub-optimal \acs{ETA}.
    The CPU computation uses one thread per meshlet and was measured on an AMD Ryzen 9 7950X (16C/32T).    
    }\label{tab:StripifyTable}    
    {
    \centering    
    \footnotesize
\begin{tblr}[b]{lccc}
    \toprule                                        
                     & \textbf{Gurobi} & \textbf{SCIP}       & \acs{ETA}\\
\midrule                                    
\textbf{computation time}     & 122.46\,s    & 8,122.56\,s & 10.52\,s   \\
\textbf{\acs{GTS} render time} & 1.58\,ms     & 1.58\,ms    & 0.59\,ms \\
\textbf{\acs{GTS} restarts}    & 123,784      & 123,784     & 185,480\\
\textbf{degenerate triangles} & 2,132      & 495,136     & 741,920\\
\textbf{additional meshlets}  & 1 / 12412   & 123  & 4 / 75,413\\
\bottomrule                                    
\end{tblr}


    }
\end{table}


\begin{figure}
    \includesvg[inkscapelatex=true]{figs/PDFTexFigure}
    \caption{Vector graphics with fonts rendered by latex.}\label{fig:latexfonts}
\end{figure}

\begin{figure}
    \includesvg[inkscapelatex=false]{figs/SVGFigure}
    \caption{Vector graphics with fonts directly taken from the file.}\label{fig:svgfonts}
\end{figure}

\begin{figure}
    \centering
    \includegraphics[width=\columnwidth]{imgs/img.png}
    \caption{Pixel graphics.}\label{fig:pixel}
\end{figure}

Tab.~\ref{tab:StripifyTable} and Tab.~\ref{tab:ListOfSymbols} compare our optimal 
method achieved with our 
\ac{MILP} 
of Sec.~\ref{Sec:MainPart} with Gurobi and SCIP against the sub-optimal strip of our 
\ac{ETA} 
implementation.
See Figs.~\ref{fig:latexfonts}, Figs.~\ref{fig:svgfonts}, and Figs.~\ref{fig:pixel} for different types of figures.
Here we test the acronyms.
First time we use a \ac{GPU}.
Next time, we use a \ac{GPU}.
Now, we even have multiple \acp{GPU}.

\section{Conclusion and Future Work}\label{Sec:ConclusionAndFutureWork}
\lipsum[1-16]
 
% Appendix
\begin{appendix}
\input{./secs/D0appendix}
\end{appendix}


  \backmatter  
  % Appendix
  \IfFileExists{./secs/D0appendix.tex}{    
    \appendix    
    \input{secs/D0appendix}
  }{}
  
  % Bibliography
  \printfeifbibliography
  \bibliography{common/bibliography}

  \printfeifglossary  
  
  \MakeDeclarationOfHonor
\end{document}